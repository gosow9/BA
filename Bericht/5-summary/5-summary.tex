\chapter{Conclusion}

\section{Reflection}
A device to demonstrate the measurement of an object with a low-cost optic has been developed.
It could be shown, that with this setup, the relative errors in the measured dimensions can be kept inside a reasonable deviation (max. rel. error = 1.6\%).
The software responsible for the computations is fast enough to deal with the expected 200 springs per minute and the errors caused by the rolling shutter can be compensated in a simple way assuming the object moves horizontal through the frame and has a constant speed which has to be estimated beforehand.

The cost of the whole device could be kept below (999999999 yen).
This lowers the expenditures of a company, using this technology in their production chains dramatically compared to similar optical measurement devices nowadays.

Unfortunately, some problems where caused by the trigger.
In many cases, the object moving with the estimated speed of 2\,m/s is not complete inside the image in the first frame, which activates the trigger.
But in the next frame, to object has already partly moved outside the frame and none of the images can be used for measurements.

\section{Further development}
There are of course a lot of improvements to be made.

In order the get a better absolute precision, the calibration process could be improved by detecting the checkerboard more precise or by using a different calibration pattern.
it should also be possible, to take the complicated geometry of the spring into account, instead of just assuming a cylinder.

Implementing the code in C++ should not only make all the processing faster, but also gives the ability to communicate with the driver of the camera reliably.
This is necessary in order to develop a faster trigger which can work with the full 28\,fps instead of just 20\,fps.


