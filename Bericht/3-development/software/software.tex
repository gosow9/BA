\section{Software}
To measure an object with a camera, the software has to perform lots of different tasks in a certain period of time.
First off, it needs to take the non linear distortions of the camera into account.
After that, some sort of trigger is needed, to check the frames for objects and pass a frame on, if an object is within the frame.
Now, the software needs to detect a known pattern on the plane where the object lies.
With the known distances of the pattern, a pixel per metric unit can be calculated.
This unit describes, how many pixels in this distance from the camera fit into one metric unit (in this case mm).
This pattern can also be used to calculate a transformation matrix which compensates for slight angular errors resulting from an imprecise mounting of the camera.
After all these steps, the object has to be recognized.
Finally, the dimensions of the object have to be estimated.

All these steps are described in more detail in the following sections.


\subsection{Initialization}
gobal variables, mapx mapy, init mode

\subsection{Trigger}

\subsection{Edge detection}
morphology

\subsection{Pattern recognition}
ppm, transformation

\subsection{Object recognition and geometry estimation}
